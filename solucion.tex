\documentclass[spanish, a4paper]{article}
\setlength{\parindent}{0em}
\usepackage{a4wide}
\usepackage{amsmath, amscd, amssymb, amsthm, latexsym}
\usepackage[spanish,activeacute]{babel}
\usepackage{enumerate}
\usepackage[utf8]{inputenc}
\usepackage{amsmath}
\newcommand{\tupla}[2]{#1 \times #2}
\input{Algo1Macros}

\setlength{\parskip}{0.1em}
\usepackage{caratula} % Version modificada para usar las macros de algo1 de ~> https://github.com/bcardiff/dc-tex



\begin{document}
\newcommand{\toroide}{\textit{toroide}}
\newcommand{\bin}{\ensuremath{\ent \times \ent}}

\titulo{TP de Especificación}
\fecha{\today}
\materia{Algoritmos y Estructuras de Datos I}
\grupo{Grupo: 15}

% CAMBIAR INTEGRANTES
\integrante{Mauricio David Toranzo}{63/20}{david-toranzo-@hotmail.com}
\integrante{Matias Federico Sarmiento}{741/18}{mfsarmiento@gmail.com}
\integrante{Victor Manuel Asmad Murga}{760/19}{victorasmad2@gmail.com}
\integrante{Marco Antonio Avila Tapia}{412/20}{marco6267@hotmail.com}

\maketitle

\section{Ejercicios - Primera Parte}



%ESCRIBIR  SOLUCIONES AQUÍ

\begin{ejercicio}[:]
\textbf{pred esValido(t: $\toroide$)}


\end{ejercicio}

\begin{ejercicio}[:]
\textbf{pred toroideMuerto(t: $\toroide$)}

\end{ejercicio}


\begin{ejercicio}[:]
\textbf{pred posicionesVivas(t: $\toroide$, vivas : $\TLista{\tupla{\ent}{\ent}}$)}

\end{ejercicio}

\begin{ejercicio}[:]
\textbf{aux densidadPoblacion(t: $\toroide$) = $\float$}

\end{ejercicio}

\begin{ejercicio}[:]
\textbf{aux cantVecinosVivos(t: $\toroide$, f: $\ent$, c: $\ent$) = $\ent$}

\end{ejercicio}


\begin{ejercicio}[:]
\textbf{pred evolucionDePosicion(t: $\toroide$, posicion : $\tupla{\ent}{\ent}$)}

\end{ejercicio}


\begin{ejercicio}[:]
\textbf{pred evolucionToroide(t1: $\toroide$, t2: $\toroide$)}

\end{ejercicio}

\section{Ejercicios - Segunda Parte}


\begin{proc}{evolucionMultiple}{in t: $\toroide$, in k: $\ent$, out result: $\toroide$}{}
    \pre{esValido(t) \wedge k > 0}
    \post{|t| = |result| \yLuego |t[0]| = |result[0]| \wedge esKesimaEvolucion(t,k,result)}

\end{proc}


\begin{proc}{esPeriodico}{in t: $\toroide$, inout p: $\ent$, out result: $\bool$}{}
    \pre{esValido(t) \wedge p=P_{0}}
    \post{P_{0} > 0 \wedge
        result = \True \Iff (\exists k:\ent)(k > 0 \implica \ (esKesimaEvolucion(t,k,t) \wedge p = k))}

\end{proc}

\begin{proc}{primosLejanos}{in t1: $\toroide$, in t2: $\toroide$, out primos: $\bool$}{}
    \pre{esValido(t1) \wedge esValido(t2)}
    \post{primos = \True \Iff (\exists k:\ent)\\
    (k>0 \implicaLuego ( (esKesimaEvolucion(t1,k,t2)) \vee (esKesimaEvolucion(t2,k,t1)) ))}

\end{proc}

\begin{proc}{seleccionNatural}{in ts: $\TLista{\toroide}$, out res: $\ent$}{}
    \pre{|ts| > 0 \wedge (\forall i : \ent)(0 \leq i < |ts| \implicaLuego esValido(ts[i]))}
    \post{
        0 \leq res < |ts| \yLuego (\forall i:\ent)(0 \leq i < |ts| \implicaLuego \\
        (\exists k,w : \ent)(k>w \wedge k>0 \wedge w>0 \wedge \\
            muerteEnTicks(ts[res],k) \geq muerteEnTicks(ts[i], w)))
    }

\end{proc}

\begin{proc}{fusionar}{in t1: $\toroide$, in t2: $\toroide$, out res: $\toroide$}{}
    \pre{esValido(t1) \wedge esValido(t2) \wedge |t1| = |t2| \wedge |t1[0]| = |t2[0]|}
    \post{|result| = |t1| \yLuego |result[0]| = |t1[0]| \implica contieneVivas(t1, t2, result)
    %(contieneToroideVivo(result, t1) \wedge contieneToroideVivo(result, t2))
}

\end{proc}

\begin{proc}{vistaTrasladada}{in t1: $\toroide$, in t2: $\toroide$, out res: $\bool$}{}
    \pre{esValido(t1) \wedge esValido(t2) \wedge |t1| = |t2| \wedge |t1[0]| = |t2[0]|}
    \post{
        res = \True \Iff esTraslado(t1, t2)
    }

\end{proc}
\clearpage
\begin{proc}{menorSuperficieViva}{in t: $\toroide$, out res: $\ent$}{}
    \pre{esValido(t) \wedge \neg toroideMuerto(t)}
    \post{\\
        (\exists ts:\TLista{toroide}) \\
        (esListaDeTraslados(ts,t) \wedge
        (\exists tMenor:\toroide) \\
        (tMenor \in ts \wedge 
        (\forall tItem \in ts)(tieneSuperficieMasChica(tMenor, t, res) \\
        )))
    }
\end{proc}

\begin{proc}{enCrecimiento}{in t: $\toroide$, out res: $\bool$}{}
    \pre{esValido(t)}
    \post{res = \True \Iff (\exists tEvo:toroide) \\
        (|tEvo| = |t| \wedge |tEvo[0]| = |t[0]| \wedge evolucionToroide(t, tEvo) \\
            \wedge 
            (\exists trasladoInicial, trasladoFinal : \toroide) \\
            (
                esTraslado(t, trasladoInicial) \wedge \\
                esTraslado(t, trasladoFinal) \wedge \\
                \neg (\exists k:\ent)(
                    tieneSuperficieMasChica(trasladoInicial, trasladoFinal, k)
                )
            )
        )
    )
}

\end{proc}

\section{Funciones y Predicados Auxiliares:}

\pred{esKesimaEvolucion}{t:toroide, k:\ent, result: toroide}{\\
    (\exists \ ts:seq<toroide>) \\
        (|ts|=k \ \yLuego \ ts[0] = t \wedge ts[k - 1]=result \
        \wedge 
        \ (\forall i : \ent) \\
            (0 \leq i < |ts|-1 \implicaLuego \ evolucionToroide(ts[i], ts[i+1]))))
}
\vspace{5mm}

\pred{muerteEnTicks}{t:toroide, k:\ent}{\\
    (\exists tm:toroide)(|tm| = |t| \yLuego |tm[0]| = |t[0]| \wedge toroideMuerto(tm)
    \yLuego esKesimaEvolucion(t,k,tm))
}
\vspace{5mm}

\pred{contieneVivas}{t1:toroide, t2:toroide, result:toroide}{\\
    (\forall i:\ent)(0\leq i < |t1| \yLuego (\forall j:\ent)(0\leq j < |t1[i]| \implicaLuego (t1[i][j] = \True \wedge t2[i][j] = \True \wedge result[i][j] = \True)))
}
\vspace{5mm}

\pred{esTraslado}{t1:toroide, t2:toroide}{\\
    (\exists k:\ent)(0 \leq k < |t1| \yLuego (\exists l:\ent)(0 \leq l < |t1[0]| \\ 
        \implicaLuego (\forall i:\ent)(0 \leq i < |t1| \yLuego (\forall j:\ent)(0 \leq j < |t1[0]| \\
        \implicaLuego (t1[(i + k) mod |t1|][(j+l) mod |t1[0]| = t2[(i + k) mod |t1|][(j+l) mod |t1[0]|
        )
        ))
    ))\\
}
\vspace{5mm}

%\pred{laMenorSuperficie}{ts:\TLista{toroide}, t:toroide, res:\ent}{\\
%    (\exists m1:\matriz{\bool})(
%        (\exists t1 \in ts)(
%            (|m| \leq |t1|) \yLuego (|m[0]| \leq |t1[0]|) \\
%            \wedge cantVivas(m1) = cantVivas(t) \\ \wedge estaContenido(m1, t1) 
%            \wedge (\forall i:\ent)(0\leq ts < |ts|-1 \\
%                \yLuego (\exists m2: \matriz{\bool})(
%                    |m2| \leq |t| \yLuego |m2[0]| \leq |t[0]| \\ \wedge estaContenido(m2, ts[i]) \wedge \\
%                    superficieTotal(m1) \leq superficieTotal(m2) \wedge superficieTotal(m1) = res
%                )
%            )
%        )
%    )
%}
%\vspace{5mm}

\pred{tieneSuperficieMasChica}{tMenor: \toroide, tComparado:\toroide, res:\ent}{\\
    (\exists matrizMenor, matrizComparada:\matriz{\bool})
    (   \\
        estaEnRango(matrizMenor, tMenor) \wedge  \\
        estaEnRango(matrizComparada, tComparado) \wedge  \\
        cantVivas(matrizMenor) = cantVivas(tMenor) \wedge  \\
        cantVivas(matrizComparada) = cantVivas(tComparado) \wedge  \\
        estaContenido(matrizMenor, tMenor) \wedge  \\
        estaContenido(matrizComparada, tComparado) \wedge  \\
        superficieTotal(matrizMenor) = res \wedge  \\
        superficieTotal(matrizMenor) \leq superficieTotal(matrizComparada) \\
    )
}
\vspace{5mm}

\clearpage

\pred{esListaDeTraslados}{ts:\TLista{toroide}, t:toroide}{\\
(\forall i:\ent)(0\leq i < |ts| \implicaLuego esValido(t) \wedge |ts[i]| = |t| \wedge |ts[i][0]| = |t[0]| \wedge esTraslado(t, ts[i]))\\
}
\vspace{5mm}

\pred{estaEnRango}{m:\matriz{\bool}, t:\toroide}{
    0 < |m| \leq |t| \wedge 0 < |m[0]| \leq |t[0]|
}
\vspace{5mm}
 
\pred{estaContenido}{m:\matriz{\bool}, tAux:toroide}{\\
    |m| \leq |tAux| \yLuego |m[0]| \leq |tAux[0]| \implica
    (\forall i:\ent)(0\leq i <|m| \yLuego (\forall j:\ent)(0\leq j < |m[0]| \implicaLuego m[i][j] = tAux[i][j]))
}
\vspace{5mm}

\aux{cantVivas}{t:toroide}{\ent}{\\
    (\sum_{i=0}^{|t|-1} (\sum_{j=0}^{|t[0]|-1} \IfThenElse {(t[i \ mod \ |t|] \ [j \ mod \ |t[0]|])}{1}{0}))
}
\vspace{5mm}

\aux{superficieTotal}{m:\matriz{\bool}}{\ent}{|m|*|m[0]|}


\section{Decisiones tomadas}

%Aquí deben agregar todo lo que hayan asumido para realizar cada ejercicio. No tienen que hablar sobre la estructura (por ejemplo justificar por qué separaron predicados/auxiliares o explicarlos). Tampoco repetir lo que dice el enunciado, pero SI cualquier asunción extra que realicen (en caso de que hagan alguna).

%Tomar en cuenta que no pueden asumir cualquier cosa, ya que podrían estar simplificando el ejercicio, consulten cualquier decisión que tomen! :)

\end{document}
