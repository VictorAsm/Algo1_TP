\documentclass[a4paper]{article}
\input{Algo1Macros}

\usepackage{a4wide}
\usepackage{amsmath, amscd, amssymb, amsthm, latexsym}
\usepackage[spanish,activeacute]{babel}
\usepackage{enumerate}

\setlength{\parskip}{0.1em}
\usepackage{caratula} % Version modificada para usar las macros de algo1 de ~> https://github.com/bcardiff/dc-tex

\newcommand{\toroide}{Toroide}

\begin{document}

\titulo{TP de Especificación}
\fecha{\today}
\materia{Algoritmos y Estructuras de Datos I}
\grupo{Grupo: 15}

% CAMBIAR INTEGRANTES
\integrante{Mauricio David Toranzo}{63/20}{david-toranzo-@hotmail.com}
\integrante{Matias Federico Sarmiento}{741/18}{mfsarmiento@gmail.com}
\integrante{Victor Manuel Asmad Murga}{760/19}{victorasmad2@gmail.com}
\integrante{Marco Antonio Avila Tapia}{412/20}{marco6267@hotmail.com}

\maketitle

\section{Ejercicios - Primera Parte}

\pred{esValido}{t : \toroide}{\\
(\forall i: \ent)
(
    (0 \leq i < |t| \land |t| \geq 3) 
        \implicaLuego 
        (|t[i]| \geq 3 \wedge |t[0]| = |t[i]|)
)\\
}
\vspace{5mm}

\pred{toroideMuerto}{t : \toroide}{\\
(\forall i: \ent )(
    (\forall j: \ent)(
        (0 \leq i <|t| \land_L 0 \leq j < |t[i]|) \implicaLuego (t[i][j] = \False)))\\
        }
\vspace{5mm}

\pred{posicionesVivas}{t : \toroide, vivas: \TLista{\ent x \ent}}{\\
    \neg toroideMuerto(t) \wedge
    |vivas| > 0 \land_L (\forall i: \ent)(
     (0 \leq i < |vivas| ) \implicaLuego \\ ((0 \leq vivas[i]_0 < |t| \land_L 0 \leq vivas[i]_1 < |t[0]|) \land_L (t[vivas[i]_0][vivas[i]_1] = \True)))\\
     }
\vspace{5mm}

\aux{densidadPoblacion}{t : \toroide}{\ent}{\\
    (\sum_{i=0}^{|t|-1} (\sum_{j=0}^{|t[i]|-1} if {(t[i][j] = \True)} \ then \ 1 \ else \ 0 \ fi)) / (|t| * |t[0]|)}
\vspace{5mm}

\aux{cantVecinosVivos}{t : \toroide, f : \ent, c : \ent}{\ent}{\\
    (\sum_{i=f-1}^{f+1} (\sum_{j=c-1}^{c+1} if {(i \neq f \land j \neq c \land (t[i\ mod\ |t|][j\ mod\ |t[0]|] = \True))} \ then \ 1 \ else \ 0 \ fi))}
\vspace{5mm}

\pred{evolucionDePosicion}{t : \toroide, posicion : \ent x\ent}{\\
    0 \leq posicion_{0} < |t| \wedge 0 \leq posicion_{1} < |t[0]| \wedge 
    \\
    \IfThenElse{t[posicion_{0}][posicion_{1}]}{
        2 \leq cantVecinosVivos(t, posicion_{0},posicion_{1}) \leq 3\\
        }{cantVecinosVivos(t,posicion_{0}, posicion_{1}) = 3}\\
}
\vspace{5mm}

\pred{evolucionToroide}{t1 : \toroide, t2 : \toroide}{\\
|t1| = |t2| \land |t1[0]| = |t2[0]| \land_L \\ (\forall i: \ent)(
    0 \leq i < |t1| \yLuego (\forall j: \ent)(
        0 \leq j < |t1[0]| \implicaLuego (evolucionDePosicion(t1, (i,j)) = t2[i][j])))\\}

\section{Ejercicios - Segunda Parte}

\begin{proc}{evolucionMultiple}{in t: $\toroide$, in k: $\ent$, out result: $\toroide$}{}
    \pre{esValido(t) \wedge k > 0}
    \post{coincideTamanio(t1, result) \wedge esKesimaEvolucion(t,k,result)}
\end{proc}


\begin{proc}{esPeriodico}{in t: $\toroide$, inout p: $\ent$, out result: $\bool$}{}
    \pre{esValido(t) \wedge p>0}
    \post{
        result = \True \Iff (\exists k:\ent)(k > 0 \implicaLuego \ (esKesimaEvolucion(t,k,t) \wedge p = k))}
\end{proc}

\begin{proc}{primosLejanos}{in t1: $\toroide$, in t2: $\toroide$, out primos: $\bool$}{}
    \pre{esValido(t1) \wedge esValido(t2)}
    \post{primos = \True \Iff (\exists k:\ent)\\
    (k>0 \implicaLuego ( (esKesimaEvolucion(t1,k,t2)) \vee (esKesimaEvolucion(t2,k,t1)) ))}
\end{proc}

\clearpage
\begin{proc}{seleccionNatural}{in ts: $\TLista{\toroide}$, out res: $\ent$}{}
    \pre{|ts| > 0 \wedge noHayToroidesInmortales(ts) }
    \post{
        0 \leq res < |ts| \yLuego toroideTardaMasEnMorir(ts, res)
    }

\end{proc}

\begin{proc}{fusionar}{in t1: $\toroide$, in t2: $\toroide$, out res: $\toroide$}{}
    \pre{esValido(t1) \wedge esValido(t2) \wedge coincideTamanio(t1, t2)}
    \post{coincideTamanio(t1, res) \wedge coincideTamanio(t2, res) \implica contieneVivas(t1, t2, result)
    %(contieneToroideVivo(result, t1) \wedge contieneToroideVivo(result, t2))
}

\end{proc}

\begin{proc}{vistaTrasladada}{in t1: $\toroide$, in t2: $\toroide$, out res: $\bool$}{}
    \pre{esValido(t1) \wedge esValido(t2) \wedge coincideTamanio(t1,t2)}
    \post{
        res = \True \Iff esTraslado(t1, t2)
    }

\end{proc}

\begin{proc}{menorSuperficieViva}{in t: $\toroide$, out res: $\ent$}{}
    \pre{esValido(t) \wedge \neg toroideMuerto(t)}
    \post{\\
        (\exists \ ts:\TLista{toroide}) (esListaDeTraslados(ts,t) \wedge (\exists \ tMenor:\toroide) \\
        (tMenor \in ts \wedge 
        (\forall \ tItem \in ts)(tieneSuperficieMasChica(tMenor, t, res))))
    }
\end{proc}

\begin{proc}{enCrecimiento}{in t: $\toroide$, out res: $\bool$}{}
    \pre{esValido(t)}
    \post{res = \True \Iff (\exists tEvo:toroide) \\
        (coincideTamanio(t, tEvo) \wedge evolucionToroide(t, tEvo) \\
            \wedge evolucionTieneSuperficieMayor(tEvo, t)
        )
    )
}

\end{proc}

\section{Funciones y Predicados Auxiliares:}

\pred{coincideTamanio}{t:toroide, tAux:toroide}{
    |t| = |tAux| \yLuego |t[0]| = |tAux[0]|
}
\vspace{5mm}

\pred{esKesimaEvolucion}{t:toroide, k:\ent, result: toroide}{\\
    (\exists \ ts:seq<toroide>) \\
        (|ts|=k \ \yLuego \ ts[0] = t \wedge ts[k - 1]=result \
        \wedge 
        \ (\forall i : \ent) \\
            (0 \leq i < |ts|-1 \implicaLuego \ evolucionToroide(ts[i], ts[i+1]))))
}
\vspace{5mm}

\pred{noHayToroidesInmortales}{ts: \TLista{\toroide}}{\\
    (\forall i:\ent)(0 \leq i < |ts| \implicaLuego esValido(ts[i]) \wedge (\exists k:\ent)(k > 0 \wedge muerteEnTicks (ts[i], k))
}
\vspace{5mm}

\pred{toroideTardaMasEnMorir}{ts: $\TLista{\toroide}$, res: $\ent$}{\\
    (\exists k:\ent)(k>0 \implicaLuego
        (\forall i:\ent)(0 \leq i < |ts| \yLuego i \neq res \implica \\
        (\exists w: \ent)(k > w \wedge w>0 \wedge \\
            muerteEnTicks(ts[res],k) \wedge muerteEnTicks(ts[i], w)))
    )
}
\vspace{5mm}

\pred{muerteEnTicks}{t:toroide, k:\ent}{\\
    (\exists tm:toroide)(coincideTamanio(tm, t) \wedge toroideMuerto(tm) \wedge esKesimaEvolucion(t,k,tm))
}
\vspace{5mm}

\pred{contieneVivas}{t1:toroide, t2:toroide, result:toroide}{\\
    (\forall i:\ent)(0\leq i < |t1| \yLuego (\forall j:\ent)(0\leq j < |t1[i]| \implicaLuego (result[i][j] = \True \Iff (t1[i][j] = \True \wedge t2[i][j] = \True))))
}
\vspace{5mm}

%\pred{esTraslado}{t1:toroide, t2:toroide}{\\
%    (\exists k:\ent)(0 \leq k < |t1| \yLuego (\exists l:\ent)(0 \leq l < |t1[0]| \\ 
%        \implicaLuego (\forall i:\ent)(0 \leq i < |t1| \yLuego (\forall j:\ent)(0 \leq j < |t1[0]| \\
%        \implicaLuego (t1[(i + k) mod |t1|][(j+l) mod |t1[0]| = t2[(i + k) mod |t1|][(j+l) mod |t1[0]|
%        )
%        ))
%    ))\\
%}
\pred{esTraslado}{t1:toroide, t2:toroide}{\\
    (\exists x,y :\ent)(0\leq x < |t1| \yLuego 0 \leq y < |t1[0]| \implicaLuego \\
        (\forall i,k :\ent)(0 \leq i,k < |t1| \yLuego 
            (\forall j,l: \ent)(0 \leq j,l < |t1[0]| \implicaLuego t2[i][j] = t1[(i+x) mod |t|][(l+y) mod |t[0]|]
            )
        )
    )
}
\vspace{5mm}

%\pred{laMenorSuperficie}{ts:\TLista{toroide}, t:toroide, res:\ent}{\\
%    (\exists m1:\matriz{\bool})(
%        (\exists t1 \in ts)(
%            (|m| \leq |t1|) \yLuego (|m[0]| \leq |t1[0]|) \\
%            \wedge cantVivas(m1) = cantVivas(t) \\ \wedge estaContenido(m1, t1) 
%            \wedge (\forall i:\ent)(0\leq ts < |ts|-1 \\
%                \yLuego (\exists m2: \matriz{\bool})(
%                    |m2| \leq |t| \yLuego |m2[0]| \leq |t[0]| \\ \wedge estaContenido(m2, ts[i]) \wedge \\
%                    superficieTotal(m1) \leq superficieTotal(m2) \wedge superficieTotal(m1) = res
%                )
%            )
%        )
%    )
%}
%\vspace{5mm}

\pred{tieneSuperficieMasChica}{tMenor: \toroide, tComparado:\toroide, res:\ent}{\\
    (\exists matrizMenor, matrizComparada:\matriz{\bool})
    (   \\
        estaEnRango(matrizMenor, tMenor) \wedge  \\
        estaEnRango(matrizComparada, tComparado) \wedge  \\
        cantVivas(matrizMenor) = cantVivas(tMenor) \wedge  \\
        cantVivas(matrizComparada) = cantVivas(tComparado) \wedge  \\
        estaContenido(matrizMenor, tMenor) \wedge  \\
        estaContenido(matrizComparada, tComparado) \wedge  \\
        superficieTotal(matrizMenor) = res \wedge  \\
        superficieTotal(matrizMenor) \leq superficieTotal(matrizComparada) \\
    )
}
\vspace{5mm}

\pred{esListaDeTraslados}{ts:\TLista{toroide}, t:toroide}{\\
(|ts| > 0 \yLuego sinRepetidos(ts)) \implica
(\forall i:\ent)(0\leq i < |ts| \implicaLuego esValido(t) \wedge coincideTamanio(ts[i], t) \wedge esTraslado(t, ts[i]))\\
}
\vspace{5mm}

\pred{estaEnRango}{m:\matriz{\bool}, t:\toroide}{
    0 < |m| \leq |t| \wedge 0 < |m[0]| \leq |t[0]|
}
\vspace{5mm}
 
\pred{estaContenido}{m:\matriz{\bool}, tAux:toroide}{\\
    |m| \leq |tAux| \yLuego |m[0]| \leq |tAux[0]| \implica
    (\forall i:\ent)(0\leq i <|m| \yLuego (\forall j:\ent)(0\leq j < |m[0]| \implicaLuego m[i][j] = tAux[i][j]))
}
\vspace{5mm}

\pred{sinRepetidos}{ts:\TLista{\toroide}}{
    (\forall i:\ent)(0 \leq i < |ts| \implicaLuego (\forall j:\ent)((0 \leq j < |ts| \yLuego i \neq j) \implica ts[i]\neq ts[j]))
}
\vspace{5mm}

\aux{cantVivas}{t:toroide}{\ent}{\\
    (\sum_{i=0}^{|t|-1} (\sum_{j=0}^{|t[0]|-1} \IfThenElse {(t[i \ mod \ |t|] \ [j \ mod \ |t[0]|])}{1}{0}))
}
\vspace{5mm}

\aux{superficieTotal}{m:\matriz{\bool}}{\ent}{|m|*|m[0]|}
\vspace{5mm}

\pred{evolucionTieneSuperficieMayor}{tEvo:\toroide, t:\toroide}{\\
    (\exists trasladoTInicial, trasladoEvolucion : \toroide) \\
    (
        esTraslado(t, trasladoTInicial) \wedge  \\
        esTraslado(tEvo, trasladoEvolucion) \wedge \\
        \neg (\exists k:\ent)(
            tieneSuperficieMasChica(trasladoTInicial, trasladoEvolucion, k)
        )
    )\\
}

\section{Decisiones tomadas}

%Aquí deben agregar todo lo que hayan asumido para realizar cada ejercicio. No tienen que hablar sobre la estructura (por ejemplo justificar por qué separaron predicados/auxiliares o explicarlos). Tampoco repetir lo que dice el enunciado, pero SI cualquier asunción extra que realicen (en caso de que hagan alguna).

%Tomar en cuenta que no pueden asumir cualquier cosa, ya que podrían estar simplificando el ejercicio, consulten cualquier decisión que tomen! :)
Usamos la primer fila en nuestras funciones y predicados púes los toroides son matrices (todas sus filas tienen el mismo largo y sus columnas el mismo alto), por lo tanto no cambia si usamos la primer o la i-esima fila. 

\end{document}
